\documentclass{report}
\usepackage{graphicx}
\usepackage{xepersian}
\usepackage{geometry}
\settextfont[Scale=1.2]{XB Zar}
\renewcommand{\baselinestretch}{1.8}

% absolute position title
\usepackage{textpos}

% section numbering
\renewcommand{\thesection}{\arabic{section}}
\renewcommand{\thesubsection}{\thesection.\arabic{subsection}}
\renewcommand{\thesubsubsection}{\thesection.\arabic{subsection}.\arabic{subsubsection}}

\title{
\begin{normalsize}
به نام خدا
\end{normalsize}
\\[4cm]
طراحی نقشه ساختمان با استفاده از پردازش تکاملی
}
\author{علیرضا نوریان
\\
\\ \small دانشگاه علم و صنعت ایران
\\ \small noorian@comp.iust.ac.ir
}
\begin{document}
\maketitle

\tableofcontents

\begin{abstract}

\end{abstract}

\section{مقدمه}
زندگی مدرن برای بشر مزایا و البته معایب بسیار زیادی به همراه آورده است. بسیاری از مشکلات حاصل از زندگی نو ریشه‌های عمیق فرهنگی دارند و شاید پاسخهای متناسب با ارزشهای دنیای جدید، برای آنها مناسب نباشند ولی در همه موارد این حرف صحیح نیست. پروژه‌ی پیش‌رو نمونه‌ای از حل مشکلات دنیای نو با استفاده از راهکارهای آن است. به طور مختصر می‌توان گفت هدف این پروژه فراهم کردن خانه‌های عملکرگرا برای سازنده‌ی مقتصد است که البته خریدار محکوم به زندگی در دنیای جدید از آن نفع خواهد برد.
به وجود آمدن مفهموم «بازار مسکن» را می‌توان از پیامدهای این زندگی جدید دانست. بازار مسکن به عنوان یکی از بازارهای پر سود موجب رشد و برخواستن جماعت «بساز و بفروش»ها شده است و این یعنی پایین آمدن کیفیت محل سکونت، کم شدن مساحت و ... که بدون تردید تاثیر بسیار بدی روی فرهنگ جامعه گذاشته است. این پروژه با هدف فراهم کردن نقشه‌ی بهینه ساختمان از جهات عملکردی انجام شده است و البته تا رسیدن به این مقصود فاصله‌ی بسیار زیادی دارد. در واقع ورودی نرم‌افزار تهیه شده در این پروژه قیدهای ساختمان از نظر طراح آن و خروجی نقشه‌ی ساختمان است.
«تولید نقشه‌ی بهینه‌ی ساختمان» جزء مسائل بهینه‌سازی محسوب می‌شود و راه حل آن کاملا وابسته به نحوه مدل‌سازی ساختمان است. در این پروژه ارائه‌ی ساختمان به گونه‌ای انجام گرفته که مساله با روشهای «پردازش تکاملی» قابل حل باشد. در ادامه پس از بررسی کوتاهی در مورد پردازش تکاملی به تعریف مساله و پاسخ داده شده به آن می‌پردازیم.

\section{پردازش تکاملی}
\subsection{الگوریتم ژنتیک}
\subsection{روشهای چند هدفه}
\subsection{روشهای ترکیبی}

\section{تعریف مساله}
«تولید نقشه‌ی ساختمان» هدف دوردستی است که برای رسیدن به آن از راه «چینش اتاقها در زمین ساختمان» عبور کردیم. در واقع نقشه‌ی حاصل نیاز به پردازشهای بعدی برای تبدیل شدن به یک نقشه واقعی دارد. برای نمونه در خروجی‌های نرم‌افزار بهترین مکان قرار دادن «در» مشخص نیست، اگرچه امکان وجود آن در تابع ارزیابی بررسی می‌شود. فرض دیگری که برای ساده‌شدن مساله صورت گرفته، مستطیل بودن زمین زیربنا است. این فرض آسیبی به پاسخ نمی‌زند و پاسخ همچنان کامل است و امید است در نگارشهای بعدی رفع شود.

\section{نحوه بیان مساله و عملگرها}

\section{تابع ارزیابی}

\section{نتایج روشهای مختلف}

\section{نتیجه‌گیری}

\section{کارهای آینده}

\bibliographystyle{plain}
\bibliography{references}

\end{document}
