\documentclass{report}
\usepackage{graphicx}
\usepackage{xepersian}
\usepackage{geometry}
\settextfont[Scale=1.2]{XB Zar}
\renewcommand{\baselinestretch}{1.8}

% absolute position title
\usepackage{textpos}

% section numbering
\renewcommand{\thesection}{\arabic{section}}
\renewcommand{\thesubsection}{\thesection.\arabic{subsection}}
\renewcommand{\thesubsubsection}{\thesection.\arabic{subsection}.\arabic{subsubsection}}

\title{
\begin{normalsize}
به نام خدا
\end{normalsize}
\\[4cm]
طراحی نقشه ساختمان با استفاده از پردازش تکاملی
}
\author{علیرضا نوریان
\\
\\ \small دانشگاه علم و صنعت ایران
\\ \small noorian@comp.iust.ac.ir
}
\begin{document}
\maketitle

\tableofcontents

\begin{abstract}

\end{abstract}

\section{مقدمه}
زندگی مدرن برای بشر مزایا و البته معایب بسیار زیادی به همراه آورده است. بسیاری از مشکلات حاصل از زندگی نو ریشه‌های عمیق فرهنگی دارند و شاید پاسخهای متناسب با ارزشهای دنیای جدید، برای آنها مناسب نباشند ولی در همه موارد این حرف صحیح نیست. پروژه‌ی پیش‌رو نمونه‌ای از حل مشکلات دنیای نو با استفاده از راهکارهای آن است. به طور مختصر می‌توان گفت هدف این پروژه فراهم کردن خانه‌های عملکرگرا برای سازنده‌ی مقتصد است که البته خریدار محکوم به زندگی در دنیای جدید از آن نفع خواهد برد.
به وجود آمدن مفهموم «بازار مسکن» را می‌توان از پیامدهای این زندگی جدید دانست. بازار مسکن به عنوان یکی از بازارهای پر سود موجب رشد و برخواستن جماعت «بساز و بفروش»ها شده است و این یعنی پایین آمدن کیفیت محل سکونت، کم شدن مساحت و ... که بدون تردید تاثیر بسیار بدی روی فرهنگ جامعه گذاشته است. این پروژه با هدف فراهم کردن نقشه‌ی بهینه ساختمان از جهات عملکردی انجام شده است و البته تا رسیدن به این مقصود فاصله‌ی بسیار زیادی دارد. در واقع ورودی نرم‌افزار تهیه شده در این پروژه قیدهای ساختمان از نظر طراح آن و خروجی نقشه‌ی ساختمان است.
«تولید نقشه‌ی بهینه‌ی ساختمان» جزء مسائل بهینه‌سازی محسوب می‌شود و راه حل آن کاملا وابسته به نحوه مدل‌سازی ساختمان است. در این پروژه ارائه‌ی ساختمان به گونه‌ای انجام گرفته که مساله با روشهای «پردازش تکاملی» قابل حل باشد. در ادامه پس از بررسی کوتاهی در مورد پردازش تکاملی به تعریف مساله و پاسخ داده شده به آن می‌پردازیم.

\section{پردازش تکاملی}
\subsection{الگوریتم ژنتیک}
\subsection{استراتژی تکاملی}
\subsection{روشهای چند هدفه}
\subsection{روشهای ترکیبی}

\section{تعریف مساله}
«تولید نقشه‌ی ساختمان» هدف دوردستی است که برای رسیدن به آن از راه «چیدن اتاقها در زمین ساختمان» عبور کردیم. در واقع نقشه‌ی حاصل نیاز به پردازشهای بعدی برای تبدیل شدن به یک نقشه واقعی دارد. برای نمونه در خروجی‌های نرم‌افزار بهترین مکان قرار دادن «در» مشخص نیست، اگرچه امکان وجود آن در تابع ارزیابی بررسی می‌شود. فرض دیگری که برای ساده‌شدن مساله صورت گرفته، مستطیل بودن زمین زیربنا است. این فرض آسیبی به پاسخ نمی‌زند و امید است در نگارشهای بعدی رفع شود.
پس می‌توان گفت تعریف دقیق مساله: «چیدن اتاقهایی با مساحت و یا اندازه‌های مشخص در زمینی مستطیل‌شکل» است. مشخص کردن مساحت اتاقها و یا اندازه‌ی دقیق طول و عرض آنها بر عهده کاربر است، اگرچه نرم‌افزار در آینده می‌تواند به کاربر کمک کند.
همچنین برای توصیف پاسخ، همه‌ی اتاقها به شکل مستطیل فرض شده‌اند و فضای دسترسی به آنها شامل (هال و راهرو) فضای باقیمانده‌ی حاصل از کم کردن دیگر اتاقها از زیربنا محسوب می‌شوند. 

\section{نحوه بیان مساله و عملگرها}
برای حل این مساله با الگوریتم ژنتیک باید پاسخ را در قالب یک ژنوم بیان کنیم. در این پروژه هر اتاق با چهار مؤلفه فاصله افقی و عمودی از مبدا و طول و عرض آن نمایش داده می‌شود. عملگرهای جهش و ترکیب عامی وجود دارند که استفاده از آنها برای همه‌ی ژنومها ممکن است. مشکل این عملگرها مستقل بودن آنها از ژنوم است برای نمونه عملگر ترکیبی که ژنومهای والد را از نقطه‌ی خاصی قطع کرده و نیمی از آنها را با هم تعویض می‌کند به اینکه هر چهار عدد پشت سر هم مربوط به یک ویژگی هستند توجه ندارد و فرزندان این دو والد ممکن است بسیار بدتر از هر دو آنها باشند. در ادامه عملگرهایی را بررسی می‌کنیم که روی این ژنوم تعریف شده‌اند.
\subsection{ترکیب}
عملگر ترکیب برای تولید فرزند از دو والد انتخاب‌شده استفاده می‌شود. در این عملگر ابتدا از دو والد، دو فرزند مشابه آنها تولید شده و پس از آن با احتمال مشخصی هر اتاق میان دو فرزند جابجا می‌شود. برای نمونه ممکن است فرزند تولید شده آشپزخانه را از یک والد و اتاق خواب را از والد دیگر به ارث ببرد.

\subsection{جهش تعویض مکان دو اتاق}
عملگر جهش که روی فرزندان حاصل از عملیات ترکیب صورت می‌گیرد. این عملگر مرکز قرار گرفتن دو اتاق را با هم عوض می‌کند و معمولا وقتی اندازه‌ی اتاقها تفاوت چندانی نمی‌کند، پاسخهای خوبی تولید می‌کند.

\section{تابع ارزیابی}
برای ارزیابی پاسخهای تولید شده، معیارهایی وجود دارند که پاسخ را به سمت شبیه خانه شدن سوق می‌دهند. این معیارها بسیار ابتدایی و کمی هستند و ارضا شدن آنها نیز بسیار آسان است. در مقابل معیارهای فضایی ضامن ویژگی‌های کیفی‌تر هستند و کمی کردن آنها پیچیده‌تر است.

معیارهای ابتدایی در واقع با بررسی اعداد توصیف کننده اتاقها محاسبه می‌شوند. برای نمونه مساحت اتاقها با مساحتی که کاربر مشخص کرده باید تطابق داشته باشند. معیار اصلی دیگری که در این گروه قرار می‌گیرد، تداخل است. اتاقها باید حداقل تداخل ممکن را با هم داشته باشند و البته باید توجه کنیم که با ایجاد میل به کاهش تداخل از روند تکامل انتظار داریم که در خروجی بین اتاقها تداخلی وجود نداشته باشد.

\subsection{فضای دسترسی}
ارضای این معیارها کاملا وابسته به تعریفی است که از فضا می‌کنیم. در این پروژه تعریف فضا به صورت قرار دادن مستطیلهای فرضی در پاسخ صورت گرفته. در واقع الگوریتمی ابتکاری برای قرار دادن بزرگترین مستطیلهای ممکن برای پوشاندن فضای خالی تعریف شده است. مستطیلهای بدست آمده از این الگوریتم ماده خام برای انجام ارزیابی‌های فضایی هستند. بزرگترین و متناسب‌ترین مستطیل از میان این مستطیلها به عنوان هال انتخاب می‌شود و بقیه نقش راهرو و پیش‌فضا را بازی می‌کنند. در تعریف پیش‌فضا باید گفت که ورودی بعضی از اتاقها مثل حمام بهتر است که در تماس مستقیم با فضای دسترسی اصلی نباشد. پوشش فضایی به گونه‌ای انجام می‌شود که فضاهای خرد انتخاب نشده و فقط فضاهای قابل استفاده انتخاب شوند. 

\subsection{ویژگی‌های معیار خوب}


\subsection{معیارها}

\subsubsection{تداخل}
روشن است که مستطیلهای نمایانگر اتاق نمی‌توانند با هم تداخل کنند و همچنین هیچکدام از آنها نمی‌توانند از فضای نقشه خارج شوند. بررسی این شرطها در معیار تداخل و با شکل میل به کاهش تداخلها انجام می‌شود.

\subsubsection{مساحت}
سیاست اصلی بزرگ شدن فضاهای دسترسی ارزشمند و از همه مهمتر فضای دسترس اصلی (هال) است. نکته‌ی دیگر کم شدن تعداد فضاهای مربوط به راهرو و پیش‌فضاست. در واقع کم شدن این فضاها به معنی بزرگ شدن فضای دسترسی اصلی و تمیز شدن آن است.

\subsubsection{دسترسی}
هر کدام از اتاقها باید از طریق فضای دسترسی قابل دستیابی باشند. دسترسی در یک ساختمان یعنی تماس میان دو فضا به اندازه‌ی حداقل یک «در» که این دو را به هم متصل کند. به طور کمی‌تر می‌توان گفت باید فاصله‌ی میان اتاق تا نزدیکترین مستطیل پوشاننده حداقل باشد.

\subsubsection{نور}
اضلاع مختلف در یک ساختمان بهره‌های متفاوتی از نور دارند. برای نمونه در خانه‌های معمول شهری بعضی از اتاقها نور مستقیم و بعضی نور آسمان (ضعیف) دریافت می‌کنند. از طرفی نیاز اتاقها نیز با هم متفاوت است. برای نمونه هال و اتاقهای خواب به نور بیشتری نیاز دارند و در مقابل سرویسها و حمام نیازی به نور ندارند. کمی کردن این معیار هم نیاز به تقریبهای مهندسی دارد. 

\section{پیاده‌سازی}
هسته‌ی اجرا کننده فرایند تکامل در این نرم‌افزار در چهارچوب ParadisEO نوشته شده است. این چهارچوب امکانات بسیار زیادی را در اختیار توسعه دهنده قرار می‌دهد و از چهار بخش اصلی تشکیل شده است ...

\section{نتایج روشهای مختلف}
در مدت انجام این پروژه روشهای زیر برای رسیدن به مجموعه جواب خوب و متنوع آزمایش شدند که در ادامه نتایج هر روش را بررسی می‌کنیم.

\subsection{الگوریتم ژنتیک}
اجرای الگوریتم ژنتیک با عملگرهای تعریف‌شده نتایج چندان جالبی به همراه ندارد. در این حالت اجرای الگوریتم نمی‌تواند تابع ارزیاب پیچیده (شامل همه‌ی موارد گفته شده) را ارضا کند و ادامه‌ی فرایند تکامل منجر به تولید پاسخ نمی‌شود. نتایج اجرا با نسخه‌ی ساده شده‌ی تابع ارزیاب (شامل مساحت و تداخل) منجر به تولید خروجی‌های نسبتا قابل قبولی می‌شود که البته چندان متنوع نیستند.

\subsection{استراتژی تکاملی}
اجرای این روش بدون استفاده از عملگرهای تعریف شده مخصوص مساله موجب تولید جواب بسیار قابل قبول شد. این نتیجه ممکن است کمی عجیب به نظر برسد و ضرورت تعریف عملگرهای مناسب را زیر سوال ببرد. نتایج و بررسی‌های بعدی نشان داد که اجرای این روش برای رسیدن به پاسخ، عملا منجر به اجرای یک جستجوی موضعی می‌شود و مستقل از تعداد جمعیت آغاز کننده فرایند تکامل، این فرایند به یک پاسخ بهینه‌ی موضعی همگرا می‌شود. این نکته‌ی بسیار مهم باعث شد که گزینه‌ی جستجوی موضعی را مورد بررسی قرار دهیم.

\subsection{روشهای چند هدفه}
اگرچه انتظار از اجرای این روشها رسیدن به تنوع بالا در جواب است، اما در این محله نیز خروجی‌ها تشابه زیادی به اجرای الگوریتم ژنتیک ساده دارند. البته با استفاده از تابع ارزیاب ساده تنوع جوابها در این روش تا حدی بهتر است.
نکته‌ی بسیار مهم در این روش و روش الگوریتم ژنتیک معمولی آن است که با افزایش تعداد فرزندان تولید شده و همچنین بالا بردن نقش عملگر جهش می‌توان به خروجی‌های بهتر و البته تقریبا همگرا دست یافت. ناگفته پیداست که این عوامل فرایند تکامل را به جستجوی محلی تبدیل می‌کنند.

\subsection{روشهای ترکیبی}


\section{نتیجه‌گیری}

\section{کارهای آینده}

\bibliographystyle{plain}
\bibliography{references}

\end{document}
