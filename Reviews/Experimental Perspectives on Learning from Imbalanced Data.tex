\documentclass{article}
\usepackage{graphicx}
\usepackage{xepersian}
\usepackage{geometry}
\settextfont[Scale=1.2]{XB Zar}

% section numbering
\setcounter{secnumdepth}{3}
\renewcommand{\thesection}{\arabic{section}}
\renewcommand{\thesubsection}{\thesection.\arabic{subsection}}
\renewcommand{\thesubsubsection}{\thesection.\arabic{subsection}.\arabic{subsubsection}}

\title{ 
\begin{normalsize} به نام خدا \end{normalsize}
\\[7cm]
نقد مقاله جنبه‌های عملی یادگیری دادگان نامتوازن
\\[3cm]
}
\author{علیرضا نوریان
\\
\\ \small دانشگاه علم و صنعت ایران
\\ \small noorian@comp.iust.ac.ir
}

\begin{document}
\maketitle

\section{خلاصه}
مواجهه با دادگان نامتوازن و رده‌بندی آنها چالشهای مختص خود را دارد. نمونه‌برداری از جمله روشهای رده‌بندی این داده‌ها است و شکلهای زیادی از آن تا کنون معرفی شده. پشتوانه‌ی نظری و آزمایشهای تجربی هر کدام از این شکلها نشان از عملکرد خوب آنها دارد. در این مقاله 35 مجموعه داده که میزان عدم توازن در آنها متفاوت است با  ۳۱ روش نمونه‌برداری شده و به ۱۱ تابع یادگیری ورودی داده شده‌اند. منظور از ۳۱ روش نمونه‌برداری در واقع ۷ روش با استفاده از پارامترهای متخلف به همراه روش عدم انجام نمونه‌برداری است. همچنین برای ارزیابی نتایج بدست آمده از معیارهای متعددی استفاده شده که هرکدام نتایج را از جنبه‌ای مورد بررسی قرار می‌دهند.
روشهای استفاده شده برای نمونه‌برداری بر هم برتری مطلق ندارند یا حداقل از نظر ریاضی برتری میان آنها اثبات نشده است. از این رو تلاش برای پیدا کردن بهترین روش نمونه‌برداری برای دادگان با میزان عدم توازن مختلف در هنگام استفاده از روشهای یادگیری مختلف هدفی است که تا حد زیادی این پژوهش دنبال شده و تا حدی بدست آمده است. جداول ۷، ۸، ۹ و ۱۰ به وضوح نشان می‌دهندکه بهترین روش نمونه‌برداری برای توابع یادگیری متفاوت چیست و هر کدام از این جدولها از معیار متفاوتی برای تعریف بهترین استفاده می‌کنند این جدولهای می‌توانند در بسیاری از پژوهشهای بعدی مورد استفاده قرار گیرند.
در زمینه‌ی پیاده‌سازی نیز می‌توان به استفاده از کتابخانه‌ی WEKA برای پیاده‌سازی انواع روشهای یادگیری از جمله درخت تصمیم اشاره کرد.

\section{نقد}
اجرای فرایند رده‌بندی با جایگشتهای مختلف الگوریتم‌های یادگیری، دادگان و روشهای نمونه‌برداری کاری بسیار سنگین و البته قابل تقدیر است که در بسیاری از تحقیقات بعدی در این زمینه و حتی در پژوهشهای کاربردی که منحصر به استفاده از تعداد کمی از این الگورتیم‌ها و روشهای نمونه برداری است، می‌تواند مورد استفاده قرار گیرد.
این مقاله هم مشابه بسیاری از مقالات دیگر نقصهایی دارد. بعضی از نقصها جزئی و کم اهمیت هستند. برای نمونه مرجع الگوریتمهای RIPPER و LR تعیین نشده است و از توضیح مختصر در مورد آنها نیز دریغ شده است. همچنین شاید بهتر بود معیارهای مورد استفاده به شکل دقیق‌تری معرفی شوند. اما اشکالات اساسی‌تری نیز بر این پژوهش وارد است.
نتایج بدست آمده در این تحقیق بسیار زیاد و قابل بحث هستند ولی اصلا تحلیلی در مورد آنها انجام نشده است. هر کدام از خانه‌ها و یا ستونهای جداول یاد شده، می‌تواند مورد تحلیل قرار گیرد. نکته‌ای که اهمیت این مساله را بیشتر می‌کند، دور از انتظار بودن پاسخهاست. روشهایی در مشاهدات برتری و غلبه دارند که ساده‌ترین شکل اجرای نمونه‌برداری هستند. در واقع تولید و یا حذف نمونه به صورت اتفاقی در بیشتر موارد از دیگر روشهای بهتر عمل کرده‌اند و این در حالی است که هر کدام از این روشهای بی‌تردید با مقاله‌ای معرفی شده که نتایج بسیار خوبی از آنها ارائه کرده است. 
البته عجیب بودن این مشاهدات نشان از اشتباه بودن آنها ندارد، بلکه نشان از اهمیت تحلیل آنها دارد و تنها دلیل قابل قبول واگذاری این تحلیلها به تحقیقات بعدی است.

\end{document}
